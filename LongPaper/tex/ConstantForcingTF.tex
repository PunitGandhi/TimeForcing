\documentclass[../main/TimeForcingSHE.tex]{subfiles} 
\begin{document}


\section{Patterns and spatial localization with a constant forcing }
NEED HELP WITH THE WORDING HERE

In this section, we describe the structure and some relevant properties of the  solutions of SHE (Eq.~\ref{eq:SH}) with a constant forcing.   Because SHE can be written in terms of the variation of a Lyaponov functional (we will refer to this as the ``Free Energy" of the system)
\begin{equation}
\mathcal{F}[u]=-\frac{1}{L}\int_{-L/2}^{L/2}\tfrac{1}{2}r u^2-\tfrac{1}{2}\left[(1+\partial_x^2)u\right]^2+M[u] \;dx
\end{equation}
where $\delta M/\delta u=N$, the solution will approach a steady-state in time that corresponds to a local minimum of this free energy if given appropriate boundary conditions (NEED TO MAKE THIS MORE PRECISE).  If we consider the space of steady-state solutions, we find that a periodic solution $u_p$ ($u_p(x)=u_p(x+2\pi)$), is formed from a bifurcation at $r=0$ where the trivial solution $u_0=0$ changes stability.  On a finite domain, the domain size and boundary conditions will determine a series of periodic solutions of differing period that emerge from the trivial branch for $r>0$ as it becomes more and more unstable.   We will focus on the case that the periodic branches  emerge subcritically and the trivial branch becomes unstable in time for $r>0$.  

\subsection{Structure of stationary solutions in and out of the pinning region}
For a suitable choice of $b$, there exists a Maxwell point for a particular value of the forcing ($r=r(M)$) where the energy ($\mathcal{F}[u_p]=\mathcal{F}[u_0]=0$) and a pinning region around it where the energy between the trivial and periodic states is sufficiently close that stable localized states formed from fronts between the two can also exist.  A bifurcation diagram of the steady state solutions (Fig.~\ref{fig:BurkeSHE}, from J. Burke et et al NEED TO REPLACE WITH MY OWN) or Fig.~\ref{fig:SHEsnaking1} shows the trivial state, a periodic state, and localized solution branch within the pinning region .  
\FIGshesnaking
For our choice of parameter $b=1.8$ and a domain size $L=80\pi$ (e.g. 40 characteristic wavelengths), the trivial solution is stable for $r<0$ and becomes unstable as as the periodic solution with 40 periods ($P_{40}$) is created through a bifurcation at $r=0$.  As we are looking at the subcritical case, we see a saddle-node bifurcation of the periodic branch where it gains stability at $SN_{P}$.  Thus we have only a stable trivial solution for $r<r(SN_{P})\approx -0.3744$. At this point, a stable (and unstable) periodic solution is created but is energetically unfavorable to the trivial state.  For $r(E_-)<r<r(E_+)$, we have a zoo of localized solutions (including an entire sequence of stable localized solutions on each snaking branch) that exist in addition to the stable trivial and periodic solutions.  We note that within this region, $r(M)$ indicates the transition from the trivial state being energetically favorable to the periodic state becoming energetically favorable.  Between $r(E_+)$ and $r=0$, we again have only the periodic solution and the trivial solution as stable  but with the periodic solution now more energetically favorable.   Finally, for $r>0$, the trivial solution looses stability and only the periodic solution remains as stable.  We note that other stable solutions exist (e.g. the flat, nonzero solutions created at the transcritical bifurcation at $r=1$) but that they have not been found to play a role in our region of interest with our current choice of parameters. 
\FIGshesnakingONE
(NEED TO INCLUDE DEFINITIONS OF FRONT VARIABLE AND ALSO INCLUDE A GRAPH OF THE STATIONARY SOLUTIONS IN AMPLITUDE FRONT SPACE)

\subsection{Depinning of localized solutions outside of the pinning region}
The dynamics solutions near steady-state have also been considered.  Burke and Knobloch \cite{burke2006} have shown that near the pinning region (e.g. for $r=r(E_{\pm})\pm\delta$ where $\delta <<1$), a localized solution that was stable at the edge of the pinning region will move towards the more energetically favorable of the trivial and the periodic state at a constant rate.  Above the snaking region, for example, a localized solution will nucleate periods of the pattern in quick bursts with some longer transition time $T_{\text{nuc}}\propto \delta^{-1/2}$ in between each nucleation event.  

An numerical example of this on a domain of 40 periods of the characteristic wavelength is shown in Fig.~\ref{fig:nucleation1}.  A localized solution that is stable for $r=r(E_+)$ is initialized above the snaking region (e.g. $r=r(E_+)+ ?\delta?$) and allowed to grow until it fills the domain. 
\FIGnucleationONE
We can see from Fig.~\ref{fig:nucleation} (NEED A BETTER FIGURE HERE) that the time from one nucleation event to the next is approximately ??, though the last nucleation event seems to take a bit longer.  %As the transition time $T_{\text{nuc}}$ for this case is on the same order as the time over which the nucleation event occurs, we have likely reached the limit of applicability of the theory described above.

\FIGnucleation

We show the results of simulations  for our parameters (Fig.~\ref{fig:nucleation}) to numerically confirm the $T_{nuc}\propto \delta^{-1/2}$ law  and locate the region of validity (INCLUDE LINE FROM BURKE CALCULATION ON GRAPH TO COMPARE).  We also show a graph of the front speed (calculated as $2 \pi / T_{nuc}$) as a function of $r_0$ for the constant forcing case.   It is clear from these graphs, that the front moves faster to the left of pinning region in the parameter regime we are looking at.  We also see a transition occur near $r=r(SN_P)$ where the solution no longer decays period by period, but begins to decay by an overall amplitude decrease to the trivial state.  Figure~\ref{DecayTransition} compares the decay of a large but localized solution inititialized on a 160 period domain just above and below $r(SN_P)$. (I NEED TO REDO THESE WITH THE SOLUTION SCALED TO THE AMPLITUDE OF THE PERIODIC SOLUTION AND MAKE EPS FILES OF THEM)
\FIGdecaytransition


We also note that beyond $\delta$ of about 0.036, the decays of periods of the solutions to the left of the pinning region are no longer clearly separated by a period of slow change in the solution.  Beyond this value, instead of depinning, the solution approaches the trivial stat by an overall amplitude decay.


\subsection{Eckhaus instability and connection of snaking branch to different period periodic branch}
\FIGeckhausbifurcation
 A zoom in of the region around the bifurcation at $r=0$ (Fig.~\ref{fig:EckhausBifurcation}a) shows the bifurcation reveals a second periodic branch with 39 periods on the domain ($P_{39}$) emerging from the trivial solution at a slightly positive r.   The snaking branch, which forms at a secondary bifurcation on $P_{40}$ (Fig.~\ref{fig:EckhausBifurcation}b), actually reconnects to $P_{39}$ (Fig.~\ref{fig:EckhausBifurcation}c).  Furthermore, the depinning solution shown Fig.~\ref{fig:nucleation} grows to a solution containing 39 periods even though the simulation was initialized with a solution on the snaking branch that emerges from $P_{40}$.  This phenomenon has been explained \cite{bergeon2008} in terms of the Eckhaus instability for the case of the steady state solutions.  Intuitively, the wavelength of the localized solution is slightly longer than the characteristic wavelength of the solution because the more energetically favorable periodic state wants to expand into the trivial state.  As the localized solution grows to a domain filling size, it finds that there isn't enough room to nucleate the 40th wavelenth of the solution.  Because we are outside of the region that is Eckhaus stable, the wavelenth instead grows to fill the domain with 39 periods. 

\FIGdefectbranch
In addition, there is a branch of stationary defect solutions that extends out into the positive forcing region (Fig.~\ref{fig:DefectBranch}).  Theses solutions have 39 full periods, but with a wavelength between the solutions on $P_{39}$ and $P_{40}$.  Thus there is not enough room to fit a 40th period and the solution is not long enough to fill the domain.   The branch is formed just above the saddle node of $P_{39}$, where the snaking branch connects(red dotted line in Fig.~\ref{fig:EckhausBifurcation}c).  We show a sample solution within the pinning region and for $r>0$ (Fig.~\ref{fig:DefectBranch}b,c).  NEED TO CHECK STABILITY AND, IF UNSTABLE,  WHICH PERIODIC SOLUTION ATTRACTS IT. Note that solutions have been translated to put the defect in the center (we assume periodic boundary conditions), but the front position $f$ has be calculated assuming the defect is on the edge of the domain.  One final comment about these solutions is that this defect branch has the opposite parity of the snaking branch that emerges from $P_{40}$ and reconnects to $P_{39}$ in the sense that, with the defect at the edge of the domain, the center point is a minimum instead of a maximum.


\subsection{Description of simple toy model of SHE and nucleations??}




\end{document}
\documentclass[../main/TimeForcingSHE.tex]{subfiles} 
\begin{document}


\section{Constant forcing in time }

Here we will describe the structure and some relevant properties of the steady-state solutions of SHE with a constant forcing.   Because Eq.~\ref{} can be written in terms of the variation of a free energy (or Lyaponov functional)
\begin{equation}
\mathcal{F}[u]=-\int_{-L/2}^{L/2}\tfrac{1}{2}r u^2-\tfrac{1}{2}\left[(1+\partial_x^2)u\right]^2-M[u] \;dx
\end{equation}
where $M'[u]=N[u]$, the solution will approach a steady state in time on the periodic domain that corresponds to a local minimum of this free energy.  If we consider the space of steady-state solutions on a periodic domain, we find that a periodic solution $u_p$ ($u_p(x)=u_p(x+2\pi)$), is formed from a bifurcation at $r=0$ where the trivial solution $u_0=0$ changes stability.  We will focus on the case that the periodic branch  emerges subcritically and the trivial branch becomes unstable in time for $r>0$.  For a suitable choice of $b$ ($b=1.8$, for example), we this bifurcation structure along with a pinning region that forms around the Maxwell point (\mathcal{F}[u_p]=\mathcal{F}[u_0]=0$) where the energy between the trivial and periodic states is sufficiently close that stable localized states formed from fronts between the two can also exist.  A bifurcation diagram of the steady state solutions (Fig.~\ref{fig:BurkeSHE}, from J. Burke et et al) shows the trivial state, the periodic state, and localized solution branches within the pinning region.  

\FIGshesnaking

We see that the trivial solution is stable for $r<0$, and becomes unstable as as the periodic solution is created through a bifurcation at $r=0$.  As we are looking at the subcritical case, we see a saddle-node bifurcation of the periodic branch where it gains stability at $SN_P$.  Thus we have only a stable trivial solution for $r<r(SN_P)\approx -0.3744$. At this point, a stable periodic solution is created but is energetically unfavorable to the trivial state.  For $r(E_-)<r<r(E_+)$, we have a zoo of localized solutions (including an entire sequence of stable localized solutions on each snaking branch) that exist in addition to the stable trivial and periodic solutions.  We note that within this region, $r(M)$ indicates the transition from the trivial state being energetically favorable to the periodic state becoming energetically favorable.  We will also find it useful to define the center of the snaking region $C$, which corresponds to  the forcing parameter $r(C)\approx -0.2992$.  Between $r(E_+)$ and $r=0$, we again have only the periodic solution and the trivial solution as stable  but with the periodic solution now more energetically favorable.   Finally, for $r>0$, the trivial solution looses stability and only the periodic solution remains as stable.  We note that other stable solutions exist (e.g. the flat, nonzero solutions created at the transcritical bifurcation at $r=1$) but that they do not play a role in our region of interest with our current choice of parameters (e.g. $b=1.8$). 

In addition to understanding the structure of the stationary solutions, it will also be helpful to understand what is known about the dynamics of the localized solutions when the system is pushed just outside of the snaking region.  Burke and Knobloch \cite{burke2006} have shown that near the snaking region (e.g. for $r=r(E_{\pm})\pm\delta$ where $\delta <<1$), a localized solution will move towards the more energetically favorable of the trivial and the periodic state.  Above the snaking region, for example, a localized solution will nucleate periods of the pattern in quick bursts with some longer transition time $T\propto \delta^{-1/2}$ in between each nucleation event.  An numerical example of this on a domain of 40 periods of the characteristic wavelength is shown in Fig.~\ref{fig:nucleation}.  A localized solution that is stable for $r\approx 0.2944$ is initialized above the snaking region (e.g. $r=0.2$) and allowed to grow until it fills the domain. We note that it grows to a solution containing 39 periods, and a corresponding numerical continuation calculation produces a snaking branch of steady state solutions that emerges as a secondary bifurcation from a 40 period solution but reconnects to a 39 period solution.  This phenomenon has been explained \cite{bergeon2008} in terms of the Eckhaus instability for the case of the steady state solutions.  We can see from Fig.~\ref{fig:nucleation} that the time from one nucleation event to the next is approximately 18, though the last nucleation event seems to take a bit longer.  As the transition time $T$ for this case is on the same order as the time over which the nucleation event occurs, we have likely reached the limit of applicability of the theory described above.

\subsection{free energy and SHE}
\subsection{bistability and maxwell point}
\subsection{snakes and ladders structure within pinning region}
\subsection{front speed just outside pinning region}
\subsection{Eckhaus instability and connection of snaking branch to different period periodic branch}
\subsection{Description of simple toy model of SHE and nucleations??}




\end{document}
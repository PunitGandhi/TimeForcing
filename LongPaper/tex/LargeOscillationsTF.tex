\documentclass[../main/TimeForcingSHE.tex]{subfiles} 
\begin{document}


\section{Stability, Growth, and Decay of localized solutions under large oscillations }
\subsection{Periodic oscillations of the solution}

We now turn to oscillations of the forcing with an amplitude greater than that of the pinning region.   The system,  initialized with a localized solution that is stable for the constant forcing case at the center of the oscillation, will spend its time alternating between growth and decay.  In many cases we consider, the oscillation even takes the system to beyond $r_{SN}$ where there no longer exists a periodic solution.  It is near this value of $r$ that localized solutions in the constantly forced system decay by an overall amplitude decrease instead of by having periods decay symmetrically off of the ends.     When the growth exactly balances the decay, periodic solutions are produced that ``breathe" with the oscillation period. Given a particular oscillation amplitude, we find a range of values for the oscillation center for which these periodic solutions can exist at each value of the oscillation period.  While the general trend of the range is to narrow for increasing oscillation periods, it does not narrow monotically -- ``sweet spots" where the range is larger than the pinched regions above and below occur at regular intervals of the oscillation period.  As we increase the oscillation amplitude, the  pinching regions are stretched out in oscillation period $T_{osc}$ and shifted to less negative values of $r_0$.  




\FIGvcmcompare



	\subsubsection{stable region for $\rho=.1,.8,.6$}
	\subsubsection{$\rho$ vs $r_0$, Tosc=100}

\subsection{Growth and decay}
For the solutions that do not form periodic orbits, we can classify their growth and decay in terms of the number of periods gained or lost after one full period of the forcing



	\subsubsection{big detailed figure of nucleations per oscillation}
	\subsubsection{stability lines and avoided crossings?}
	\subsubsection{simple model interpretaion}

\subsection{some asymptotic calculations???}

\end{document}
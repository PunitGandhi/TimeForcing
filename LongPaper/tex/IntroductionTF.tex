\documentclass[../main/TimeForcingSHE.tex]{subfiles} 
\begin{document}


\section{Introduction}

\subsection{motivation for periodic time forcing in pattern formation}
\subsection{motivation for SHE}
\subsection{Description for SHE}
\subsection{Numerical Methods}
\subsection{paper outline}

%The Swift--Hohenberg equation~\cite{swift1977} (SHE) serves as a model for pattern formation in a broad range of physical systems~\cite{}.  The existence, structure, and stability of localized solutions within the SHE has been studied in great detail~\cite{burke2006,burke2007snakes,burke2007stability}.   This equation, which takes the form  
%\begin{equation}
%u_t= r u-\left(1+\partial_{x}^2\right)^2u+N[u]\label{eq:SH},
%\end{equation}
%describes the dynamics of a real field $u$ over one spatial dimension in time, where $N$ is some nonlinear function of $u$.  We have rescaled the equation so that the critical wavenumber that defines the natural wavelength of the patterned state is unity.  We will be interested in two possible choices of $N$, namely $N_{23}[u]=bu^2-u^3$ and $N_{35}[u]=b u^3-u^5$.  The strength of the linear forcing term $r$ and the strength of the quadratic/cubic nonlinearity $b$ are left as parameters of the system.  
%
%We consider the case when the forcing is no longer constant in time, namely $r\rightarrow r_0+\delta r \sin\omega t$.  Pattern formation in ecological systems that are periodically forced by the seasons or the daily variations in insolar flux are one example of a physical motivation for considering this kind of system \cite{}.  Other physical systems that might be described by such a periodically forced model include ... \cite{}.   Furthermore, oscillations that effectively create and destroy attractors have been shown to produce new ``ghost" attractors that do not exist in the time-independent system for any value of the parameter\cite{}.  While this has been done for the case of simple oscillators, the present work provides an extension of these observations to higher dimensions.
%
%We first recount the relevant details of the original Swift-Hohenberg equation before discussing some numerical results of the time-dependent forced case.  Hopefully we will also eventually add a section describing some theoretical understanding of the results we obtained.  Note that all the work done so far is for the $N_{23}$ nonlinearity and $b=1.8$.



\end{document}


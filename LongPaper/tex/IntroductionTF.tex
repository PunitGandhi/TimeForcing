\documentclass[../main/TimeForcingSHE.tex]{subfiles} 
\begin{document}


\section{Introduction}

Time-dependent forcing can be critical to the understanding of pattern formation in certain systems.  Daily and seasonal variations of insolar flux due to the Earth's rotation and orbit, for example, provide a periodic forcing for many ecological systems~\cite{kletter2012ostwald,tlidi2008vegetation,meron2012pattern,sherratt2010pattern,sherratt2005analysis}.   Furthermore,periodic and stochastic switching between systems in time  have been shown to produce new ``ghost" attractors that do not exist in either time-independent systems or the averaged system.  Studies  have observed this phenomon in coupled oscillator systems\cite{belykh2013multistable} and have used it to create patterned states from systems where they would not otherwise exist~\cite{buceta2001stationary}.   Even systems that are designed to have a constant forcing may have a time-dependence introduced by noise or other experimental considerations.  This unintentional time dependence could destabilize certain states, making them impossible to achieve experimentally.  This is particularly true of spacially localized patterns that often exist within a narrow region of parameter space.  oscillations in time  could also have the opposite effect of stabilizing a previously unstable stationary solution. Indeed, the inverted pendulum attached to a vertically oscillating base provides an example of this- the upright position of the pendulum can be stabilized if the oscillations are fast enough.  

% This study focuses on the effects of a periodic time dependence in the forcing parameter on spatially localized patterns.

The Swift--Hohenberg equation~\cite{swift1977} (SHE) serves as a model for pattern formation in a broad range of physical systems.  The existence, structure, and stability of localized solutions within the SHE has been studied in great detail~\cite{burke2006,burke2007snakes,burke2007stability}.   This equation which, in one dimension, takes the form  
\begin{equation}
u_t= r u-\left(1+\partial_{x}^2\right)^2u+N[u]\label{eq:SH},
\end{equation}
describes the dynamics of a real field $u$ over one spatial dimension in time, where $N$ is some nonlinear function of $u$.  We have rescaled the equation so that the characteristic wavenumber that defines the natural wavelength of the patterned state is unity.  Two possible choices of $N$, namely $N_{23}[u]=bu^2-u^3$ and $N_{35}[u]=b u^3-u^5$, are often considered because of the different symmetries they possess.  This study focuses on the more simple  $N_{23}$ that provides the equation with only one reflection symmetry for a given arbitrary axis ($x\rightarrow -x$, $u\rightarrow u$).  Preliminary numerical simulations indicate, however, that analgous behaviors to the results obtained here are found in the case of $N_{35}$ (SHOULD WE CHECK THAT THIS IS TRUE AND INCLUDE A STATEMENT LIKE THIS?)     The strength of the linear forcing term $r$ and the strength of the quadratic nonlinearity $b$ are left as parameters of the system.  In order to study the effects of time-dependence, we will include an addtional periodic term to the linear forcing, namely $r\rightarrow r_0+\rho \sin\omega t$, where $r_0$, $\rho$, and $\omega=2\pi/T$ define the offset, amplitude, and frequency of the oscillation. 


% Pattern formation in ecological systems that are periodically forced by the seasons or the daily variations in insolar flux are one example of a physical motivation for considering this kind of system \cite{}.  Other physical systems that might be described by such a periodically forced model include ... \cite{}.   Furthermore, oscillations that effectively create and destroy attractors have been shown to produce new ``ghost" attractors that do not exist in the time-independent system for any value of the parameter\cite{}.  While this has been done for the case of simple oscillators, the present work provides an extension of these observations to higher dimensions.

All simulations in time used periodic boundary conditions on a domain of $L=80\pi$ (e.g. 40 characteristic wavelengths), unless otherwise noted.  A 4th order exponential time differencing scheme\cite{cox2002} was used to step forward  in time while spectral methods on a grid of 1024 points were used for the spatial calculations.  In cases where a larger domain was necessary, the spatial density of grid points was kept constant.    Steady state solutions of the constant forcing case were computed by numerical continuation using AUTO~\cite{doedel1981auto}.  


We first recount the relevant details of the Swift-Hohenberg equation with a constant forcing before discussing some numerical results and theoretical analysis of the periodically forced case.   We begin by examining the  effect of small oscillations on the growth and decay of slightly  unstable localized solutions, and then move to large oscillations that extend through and beyond the pinning region of the constant forcing case.  Finally, discuss the persistence of a localized defect state that would be unstable without the oscillations before concluding with a summary of the results and an outlook on future work.

%\subsection{motivation for periodic time forcing in pattern formation}
%\subsection{Description for SHE}
%\subsection{Numerical Methods}
%\subsection{paper outline}



% We can vary the way the forcing oscillates ($r\rightarrow r_0+\rho \sin\omega t$) in 3 ways: (1) the amplitude of the oscillation, $\rho$ (2) the point about which the oscillation occurs, $r_0$ (3) the frequency with which the oscillation occurs, $\omega$.    unless otherwise noted,the simulations are initialized with a localized solution that is stable for the constant forcing ($\rho=0$) case with the given choice of $r_0$. 


\end{document}


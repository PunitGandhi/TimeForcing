\documentclass[aps,pre,longbibliography,12pt,a4paper,preprint]{revtex4-1}     
\usepackage{graphicx}
\usepackage{../main/TimeForcingSty}
\usepackage{../main/TimeforcingFigures}
%\usepackage{../main/TimeforcingFiguresEPS} %use this to compile figures from eps files


\lhead{Periodic Forcing SHE}
\chead{}
\rhead{Gandhi et al.}
%\lfoot{}
%\cfoot{}
%\rfoot{}




\pagestyle{fancy}
\begin{document}
\preprint{APS/123-QED}


\title{Periodic Forcing of the Swift-Hohenberg Equation in Time}
\author{Punit Gandhi}
 \email{punit\_gandhi@berkeley.edu}
\author{C\'edric Beaume}
\author{Edgar Knobloch}
\affiliation{Department of Physics, University of California, Berkeley CA 94720, USA}
\date{\today}

\begin{abstract}
(NEED TO WRITE A REAL ABSTRACT.)
Systems with a periodic forcing in time abound!  We use the generalized Swift-Hohenberg equation with a quadratic-cubic nonlinearity as test-bed for studying localized pattern formation in such systems with a periodic forcing in time. A sinisiodal  linear forcing is applied to the SHE in order to the examine the dependence of localization on the amplitude, oscillation period, and offset of the time-periodic forcing.  As one might expect, the region of existence of localized solutions dramatically decreases as the system is ``jiggled."  The parameter space within the pinning region of the constant forcing case, however, is partitioned into regions of growth, stability, and decay with an unexpected structure when large oscillations are applied. 

\end{abstract}

\maketitle


%Introduction
\subfile{../tex/IntroductionTF.tex}

\subfile{../tex/ConstantForcingTF.tex}

%\newpage

\subfile{../tex/PeriodicForcingTF.tex}

\subfile{../tex/SmallOscillationsTF.tex}

\subfile{../tex/LargeOscillationsTF.tex}

%\newpage

\subfile{../tex/DefectsTF.tex}

\subfile{../tex/ConclusionTF.tex}

\bibliography{../tex/TimeForcingBib}
\end{document}



%%%%%%%%%%%%%%%%%%%%%%%%%%%%%%%%%%%%%%%%%%%%%%%%%%%%%%%%%%%%%%



\documentclass[pre,preprint,superscriptaddress]{revtex4-1} 

\usepackage{graphicx}
\usepackage{hyperref}
\usepackage{amsmath}
\usepackage{amsfonts} % needed for bold Greek, Fraktur, and blackboard bold
\usepackage{amssymb}
\usepackage[margin=1in]{geometry}
\usepackage{dcolumn}
\usepackage{multirow}
\usepackage{tikz}
\usetikzlibrary{calc,patterns,decorations.pathmorphing,decorations.markings}
\usepackage{hyperref}
\usepackage{float}
\usepackage{subfig}
\usepackage{todonotes}
\usepackage{subfiles}

% TODO: remove these lines, which expand the margins (useful for comments)
\textwidth  .72\paperwidth
\hoffset -1in
\oddsidemargin .14\paperwidth
\evensidemargin .14\paperwidth
\marginparwidth .11\paperwidth


% Draft macros
\usepackage[normalem]{ulem} % for strikethrough
%\usepackage[usenames,dvipsnames]{xcolor}
%\newcommand{\TODO}[1]{\marginpar{\raggedright\scriptsize\textbf{TODO:} #1} (\textbf{TODO})}
%\newcommand{\NOTEMARG}[1]{\marginpar{\raggedright\scriptsize\textbf{NOTE:} #1} (\textbf{NOTE})}
%\newcommand{\NOTE}[1]{\marginpar{\footnotesize\textbf{NOTE}} (\textbf{NOTE: #1})}
%\definecolor{purple}{rgb}{1,0,1}


\newcommand{\eq}[1]{eq.~\eqref{eq:#1}}
\newcommand{\eqs}[2]{eqs.~\eqref{eq:#1} and \eqref{eq:#2}}
\renewcommand{\sec}[1]{section~\ref{sec:#1}}
\newcommand{\secs}[2]{sections~\ref{sec:#1} and \ref{sec:#2}}
\newcommand{\subsec}[1]{section~\ref{subsec:#1}}
\newcommand{\subsubsec}[1]{section~\ref{subsubsec:#1}}
\newcommand{\app}[1]{appendix~\ref{app:#1}}
\newcommand{\fig}[1]{figure~\ref{fig:#1}}
\newcommand{\figs}[2]{figures~\ref{fig:#1} and \ref{fig:#2}}
\newcommand{\tab}[1]{table~\ref{tab:#1}}
\newcommand{\nn}{\nonumber}




\begin{document}

\title{Time dependent forcing in the Swift-- Hohenberg equation}
\author{Punit Gandhi}
 \email{punit\_gandhi@berkeley.edu}
\author{C\'edric Beaume}
\author{Edgar Knobloch}
 \email{knobloch@berkeley.edu}
\affiliation{Department of Physics, University of California, Berkeley CA 94720, USA}
\date{\today}

\begin{abstract}
This is an update of some of the analyitic calculations attempted to provide some understanding of the effects of time-periodic forcing in the Swift-Hohenberg equation. 
\end{abstract}

\maketitle

\section{Introduction}
The equation we consider is: 
\begin{equation}
u_t= (r_0+ \rho \sin\omega t) u-\left(1+\partial_{x}^2\right)^2u+b u^2-u^3\label{eq:SHtd},
\end{equation}
which describes the dynamics of a real field $u$ over one spatial dimension in time.  We have rescaled the equation so that the critical wavenumber that defines the natural wavelength of the patterned state is unity.   The average value of the linear forcing term $r_0$, the amplitude of oscillation $\rho$, the frequency of oscillation $\omega$, and the strength quadratic nonlinearity $b$ are left as parameters of the system.



\section{Weakly nonlinear analysis}
We do asymptotic analysis in various limits of the oscillation speed and amplitude in order to understand the dynamics in places where analytic theory may be possible.  I think the fast oscillations with a large amplitude may be the most promising case because it makes the time outside the pinning region the right order of magnitude to play a nontrivial role in the dynamics.

\subsection{fast oscillations}
Here we assume that the oscillations occur at a high frequency $\omega\rightarrow\omega/\epsilon$, where $\epsilon<<1$.  We can define a fast timescale by $t\rightarrow t/\epsilon$ and rewrite Eq.~\ref{eq:SHtd} as
\begin{equation}
u_t= \epsilon\left[(r_0+ \rho \sin\omega t) u-\left(1+\partial_{x}^2\right)^2u+b u^2-u^3 -u_{\tau}\right]\label{eq:SH},
\end{equation}
where $\tau=\epsilon t$ now represents the slow timescale.  We can now assume an asymptotic series for $u$,
\begin{equation}
u=u_0+\epsilon u_1+ \epsilon^2 u_2+ ...
\end{equation}
and look at the equation order by order.  Note that in order to do this, we are implicitly assuming that $\epsilon<<\rho,r_0,b<<1/\epsilon$.    Leading order gives $(u_0)_t=0$, which has the solution $u_0(x,\tau,t)=A_0(x,\tau)$.
At order $\mathcal{O}(\epsilon)$, the equation becomes
\begin{equation}
(u_1)_t= (r_0+ \rho \sin\omega t) u_0-\left(1+\partial_{x}^2\right)^2u_0+ b u_0^2-u_0^3 -(u_0)_{\tau}\label{eq:SHfastoscu1}.
\end{equation}
The solvability condition requires that over a single period of the fast oscillation, $\int RHS dt=0$. This produces the governing equation for $A_0$:
\begin{equation}
\dot{A}_0= r_0 A_0-\left(1+\partial_{x}^2\right)^2A_0+b A_0^2-A_0^3\label{eq:SHfastoscA0}.
\end{equation}
Thus, in this limit of fast oscillation with order unity amplitude, the leading order solution is just the solution to the Swift-Hohenberg Equation.
We can now write down the form of the $\mathcal{O}(\epsilon)$ correction to the solution by integrating Eq.~\ref{eq:SHfastoscu1} and substituting in for $\dot{A}_0$ from Eq.~\ref{eq:SHfastoscA0}:
\begin{equation}
u_1=  -\tfrac{\rho}{\omega} \cos\omega t A_0(x,\tau) +A_1(x,\tau)\label{eq:SH}.
\end{equation}
 We must move onto order $\mathcal{O}(\epsilon^2)$ to find the equation that determines $A_1$.  At this order, we get
\begin{equation}
(u_2)_t= (r_0+ \rho \sin\omega t) u_1-\left(1+\partial_{x}^2\right)^2 u_1+ 2 b u_0 u_1- 3 u_0^2 u_1 -(u_1)_{\tau}\label{eq:SH}.
\end{equation}
The solvability condition for this equation becomes
\begin{equation}
\dot{A}_1 = r_0 A_1-\left(1+\partial_{x}^2\right)^2A_1+2 b A_0A_1-3A_0^2A_1\label{eq:SH}.
\end{equation}
The second order correction to the solution takes the form
\begin{equation}
u_2=  \tfrac{\rho^2}{4\omega^2} \cos2\omega t A_0(x,\tau) -\tfrac{\rho}{\omega^2} \sin\omega t \left[b A_0(x,\tau)^2-2 A_0(x,\tau)^3\right] -\tfrac{\rho}{\omega} \cos \omega t A_1(x,\tau)  + A_2(x,\tau)\label{eq:SH}.
\end{equation}
and we can find the governing equation for $A_2$ by going to order $\mathcal{O}(\epsilon^3)$.  The equation at this order is
\begin{equation}
(u_3)_t= (r_0+ \rho \sin\omega t) u_2-\left(1+\partial_{x}^2\right)^2 u_2+ 2 b u_0 u_2- 3 u_0^2 u_2 +b u_1^2-3u_0 u_1^2  -(u_1)_{\tau}\label{eq:SH}.
\end{equation}
The solvability condition is
\begin{equation}
\dot{A}_2= r_0 A_2-\left(1+\partial_{x}^2\right)^2A_2+b (2 A_0A_2+A_1^2) - 3( A_0^2A_2+A_0A_1^2)
		-\tfrac{1}{2}\left(\tfrac{\rho}{\omega}\right)^2 A_0^3  \label{eq:SH}.
\end{equation}
We can define an averaged  variable with error at order $\mathcal{O}(\epsilon^3)$ that describes the dynamics on the long time scale.  
\begin{align}
A &=\frac{1}{T}\int_0^T u_0+\epsilon u_1 +\epsilon^2 u_2 dt \\ \nonumber
	&=A_0+\epsilon A_1 +\epsilon^2 A_2
\end{align}
By summing the solvability conditions, we get the following equation for the dynamics of the averaged variable
\begin{equation}
\dot{A}= r_0 A-\left(1+\partial_{x}^2\right)^2A+b A^2-\left[1+\tfrac{1}{2}\epsilon^2\left(\frac{\rho}{\omega}\right)^2\right]A^3 +\mathcal{O}(\epsilon^3)\label{eq:SH}.
\end{equation}



\subsection{fast oscillations with a large amplitude}
We can repeat the above procedure in the case that $\rho\rightarrow \rho/\epsilon$ so that we are now dealing with fast oscillations with a large amplitude.
\begin{equation}
u_t-\rho \sin(\omega t) u= \epsilon\left[r_0 u-\left(1+\partial_{x}^2\right)^2u+b u^2-u^3 -u_{\tau}\right]\label{eq:SH},
\end{equation}
 We can again assume an asymptotic series for $u$,
\begin{equation}
u=u_0+\epsilon u_1+ \epsilon^2 u_2+ ...
\end{equation}
and look at the equation order by order.  Note that in this case, we are implicitly assuming that $\epsilon<<r_0,b<<1/\epsilon$.    Leading order gives $(u_0)_t-\rho \sin(\omega t) u_0=0$, which has the solution
\begin{equation}
u_0(x,\tau,t)=e^{-(\rho/\omega) \cos\omega t} A_0(x,\tau).\label{eq:SHfastlargeoscu0}
\end{equation}
At order $\mathcal{O}(\epsilon)$, the equation becomes
\begin{equation}
(u_1)_t-\rho \sin(\omega t) u_1= r_0 u_0-\left(1+\partial_{x}^2\right)^2u_0+ b u_0^2-u_0^3 -(u_0)_{\tau}\label{eq:SHfastlargeoscu1}.
\end{equation}
The solvability condition requires that over a single fast oscillation $\int RHS e^{(\rho/\omega) \cos\omega t}dt=0$, assuming that $u_1$ is periodic over a fast oscillation period. This produces the governing equation for $A_0$:
\begin{equation}
\dot{A}_0= r_0 A_0-\left(1+\partial_{x}^2\right)^2A_0+b I_0(\tfrac{\rho}{\omega}) A_0^2-I_0(2\tfrac{\rho}{\omega})A_0^3\label{eq:SHfastlargeA0}.
\end{equation}
where $I_0(x)$ is the modified Bessel function of the first kind evaluated at $x$. Thus the leading order solution to the problem satisfies a Swift-Hohenberg equation with modified constants on the slow time, but its amplitude oscillates over the fast time. 

The first correction to the solution can be found by integrating Eq.~\ref{eq:SHfastlargeoscu1} using the integrating factor $ e^{(\rho/\omega) \cos\omega t}$ and substituting in Eqs.~\ref{eq:SHfastlargeoscu0}~and~\ref{eq:SHfastlargeA0}  
\begin{equation}
u_1(x,\tau,t)=e^{-(\rho/\omega) \cos\omega t}\left[ \alpha_2(t) b A_0(x,\tau)^2- \alpha_3(t)A_0(x,\tau)^3 + A_1(x,\tau)\right]
\end{equation}
The time-dependent coefficients, which are periodic in the fast time, are  given by
\begin{align}
\alpha_2(t)&=\int_0^t  e^{-(\rho/\omega) \cos\omega s}-I_0(\tfrac{\rho}{\omega})ds \\ 
\alpha_3(t)&=\int_0^t  e^{-(2\rho/\omega) \cos\omega s}-I_0(2\tfrac{\rho}{\omega})ds
\end{align}
and $A_1$ is yet to be determined.

Going to order  $\mathcal{O}(\epsilon^2)$ gives
\begin{equation}
(u_2)_t-\rho \sin(\omega t) u_2= r_0 u_1-\left(1+\partial_{x}^2\right)^2u_1+ 2b u_0u_1-3u_0^2u_1 -(u_1)_{\tau}\label{eq:SH}.
\end{equation}
The solvability condition is, as before, $\int RHS e^{(\rho/\omega) \cos\omega t}dt=0$ over a fast oscillation period.  As the calculation somewhat long and tedious, a few intermediate steps will be shown in order to make it easier for me to confirm the result. Substitution of $u_0$ and $u_1$ into RHS leads to 
\begin{align}
\frac{\omega}{2\pi}\int_0^{2\pi/\omega}dt\left[r_0-\left(1+\partial_{x}^2\right)^2 -\partial_{\tau}+2 b  e^{-(\rho/\omega) \cos\omega t}A_0(x,\tau)-3e^{-2(\rho/\omega) \cos\omega t}A_0(x,\tau)^2\right]\\ \nonumber
\left( \alpha_2(t) b A_0(x,\tau)^2- \alpha_3(t)A_0(x,\tau)^3 + A_1(x,\tau)\right)=0
\end{align}
We can expand this out and perform the integration over the fast time to get
\begin{align}
\dot{A}_1=& \left(r_0 -\left(1+\partial_{x}^2\right)^2\right)A_1 + 2 b I_0(\tfrac{\rho}{\omega})A_0 A_ 1 - 3I_0(2\tfrac{\rho}{\omega}) A_0^2 A_1 \\ \nonumber
&-2\beta_2 A_0\dot{A}_0 + \beta_2\left(r_0 -\left(1+\partial_{x}^2\right)^2\right)A_0^2 + 2\eta_{23} b A_0^3 -3\eta_{24} A_0^4 \\ \nonumber
&+3\beta_3 A_0^2 \dot{A}_0 - \beta_3\left(r_0 -\left(1+\partial_{x}^2\right)^2\right)A_0^3- 2\eta_{34} b A_0^4 +3\eta_{35} A_0^5 
\end{align}
where
 \begin{align}
\beta_2&=\frac{\omega}{2\pi}\int_0^{2\pi/\omega}\alpha_2(t)dt\\
\eta_{23}&=\frac{\omega}{2\pi}\int_0^{2\pi/\omega} e^{-(\rho/\omega) \cos\omega t}\alpha_2(t)dt\\ 
\eta_{24}&=\frac{\omega}{2\pi}\int_0^{2\pi/\omega} e^{-2(\rho/\omega) \cos\omega t}\alpha_2(t)dt\\ 
\beta_3&=\frac{\omega}{2\pi}\int_0^{2\pi/\omega}\alpha_3(t)dt\\
\eta_{34}&=\frac{\omega}{2\pi}\int_0^{2\pi/\omega} e^{-(\rho/\omega) \cos\omega t}\alpha_3(t)dt\\ 
\eta_{35}&=\frac{\omega}{2\pi}\int_0^{2\pi/\omega} e^{-2(\rho/\omega) \cos\omega t}\alpha_3(t)dt
\end{align}
We can now substitute in for $\dot{A}_0$ and rearrange the terms to get
\begin{align}
\dot{A}_1=&\left[r_0 -\left(1+\partial_{x}^2\right)^2\right]A_1 + 2 b I_0(\tfrac{\rho}{\omega})A_0 A_ 1 - 3I_0(2\tfrac{\rho}{\omega}) A_0^2 A_1 \\ \nonumber
&-(2\beta_2 A_0-3\beta_3 A_0^2)\left[ r_0 A_0-\left(1+\partial_{x}^2\right)^2A_0+b I_0(\tfrac{\rho}{\omega}) A_0^2-I_0(2\tfrac{\rho}{\omega})A_0^3\right] \\ \nonumber
& + \left[r_0 -\left(1+\partial_{x}^2\right)^2\right](\beta_2 A_0^2 -\beta_3 A_0^3)
+ 2\eta_{23} b A_0^3 -(3\eta_{24}+2\eta_{34} b ) A_0^4 +3\eta_{35} A_0^5 
\end{align}
We can simplify the derivative terms:
\begin{align}
-(2\beta_2 A_0-3\beta_3 A_0^2)\left[ r_0 A_0-\left(1+\partial_{x}^2\right)^2A_0\right]+ \left[r_0-\left(1+\partial_{x}^2\right)^2\right](\beta_2 A_0^2 -\beta_3 A_0^3)= \\ \nonumber
-(r_0-1)(\beta_2 A_0^2 - 2\beta_3 A_0^3) +36\beta_3 (\partial A_0)^2\partial_x^2A_0\\ \nonumber
-(\beta_2-3\beta_3 A_0)\left( 4(\partial_xA_0)^2+6(\partial_x^2 A_0)^2+8(\partial_xA_0)(\partial_x^3A_0\right)
\end{align}
and combine further combine terms to get
\begin{align}
\dot{A}_1&=\left[r_0 -\left(1+\partial_{x}^2\right)^2\right]A_1 + 2 b I_0(\tfrac{\rho}{\omega})A_0 A_ 1 - 3I_0(2\tfrac{\rho}{\omega}) A_0^2 A_1 \\ \nonumber
&-(r_0-1)(\beta_2 A_0^2 - 2\beta_3 A_0^3) +36\beta_3 (\partial_x A_0)^2\partial_x^2A_0 \\ \nonumber
&-(\beta_2-3\beta_3 A_0)\left[ 4(\partial_xA_0)^2+6(\partial_x^2 A_0)^2+8(\partial_xA_0)(\partial_x^3A_0)\right] \\ \nonumber
&+2 b \left[\eta_{23}-I_0(\tfrac{\rho}{\omega})\beta_2\right]A_0^3 +\left[3 b I_0(\tfrac{\rho}{\omega})\beta_3+2 I_0(2\tfrac{\rho}{\omega})\beta_2-3\eta_{24}-2 b\eta_{34} \right]A_0^4+3\left[\eta_{35}- I_0(2\tfrac{\rho}{\omega})\beta_3\right]A_0^5
\end{align}



\subsection{slow oscillations}
We now consider the case that $\omega\rightarrow \epsilon \omega$ so that we are dealing with slow oscillations with amplitude of order $r_0$.  In this limit, Eq.~\ref{eq:SHtd} becomes
\begin{equation}
u_t+ \epsilon u_{\tau}=(r_0+ \rho \sin\omega \tau) u-\left(1+\partial_{x}^2\right)^2u+b u^2-u^3 \label{eq:SH}.
\end{equation}
At leading order, we just get the Swift Hohenberg equation in the fast time with the forcing term that only varies on the slow time,
\begin{equation}
\partial_t u_0=(r_0+ \rho \sin\omega \tau) u_0-\left(1+\partial_{x}^2\right)^2u_0+b u_0^2-u_0^3 \label{eq:SH}.
\end{equation}
At order $\mathcal{O}(\epsilon)$, the equation becomes
\begin{equation}
\partial_t u_1+ \partial_{\tau} u_0=(r_0+ \rho \sin\omega \tau) u_1-\left(1+\partial_{x}^2\right)^2u_1+2 b u_0 u_1-3u_0^2 u_1 \label{eq:SH}.
\end{equation}
The solvability condition tells us that $u_0$ cannot depend on the slow time. Thus $u_1$ satisfies the linearization of the SHE about the solution $u_0$.  

Going on to order  $\mathcal{O}(\epsilon^2)$, we get 
\begin{equation}
\left[(r_0+ \rho \sin\omega \tau) -\left(1+\partial_{x}^2\right)^2+2 b u_0 -3u_0^2-\partial_t \right]u_2 =\partial_{\tau}u_1-bu_1^2-3u_0u_1^2\label{eq:SH}.
\end{equation}
I don't know how to proceed analytically here.  I am also worried that this calculation does not capture the effect that we are interested because there is no assumption about the amount of time spent outside of the pinning region.  With the current assumptions, isn't the time outside of the pinning region is effectively infinite to the dynamics of the nucleation and decay processes?

\section{Front speed during depinning}
We follow Burke's calculation for the standard SHE to find the time between nucleation events just outside of the pinning region in the case of a time periodic forcing.  

\subsection{small, slow oscillations}
We will perform this calculation just  to the right of $r_+$, the right edge of the pinning region of the constant forcing case (e.g. $r\rightarrow r_++\epsilon^2\delta$).  Note that the exact same procedure could be used to find the time between decay events just to the left of the pinning region.   We will assume small (e.g. $\rho\rightarrow \epsilon^2 \rho$), slow oscillations (e.g. $\omega\rightarrow \epsilon \omega$)  so that the deviation from Burke's calculation will be small in this case.  The equation, in this limit becomes 
\begin{equation}
u_t= \left(r_++ \epsilon^2(\delta+ \rho \sin\epsilon\omega t)\right) u-\left(1+\partial_{x}^2\right)^2u+b u^2-u^3\label{eq:SH},
\end{equation}
where $r_+$ is the right edge of the pinning region when $\rho,\delta=0$.  Because we are near the pinning region, we can assume the dynamics will be slow and will define the slow timescale $\tau=\epsilon t$ and corresponding time derivative $\partial_t\rightarrow\epsilon\partial_{\tau}$.  After writing u as an asymptotic series, $u=u_0+\epsilon u_1+\epsilon^2 u_2+...$, we can write out the equation order by order in $\epsilon$. At leading order, we have
\begin{equation}
r_+u_0-\left(1+\partial_{x}^2\right)^2u_0+b u_0^2-u_0^3=0\label{eq:SH},
\end{equation}
and can thus pick $u_0$ to be a localized solution at a saddle-node bifurcation of the snaking branch.  Thus $u_0$ is stationary in time, but only marginally stable.  Going on to order  $\mathcal{O}(\epsilon)$, we get
\begin{equation}
\partial_{\tau} u_0=r_+u_1-\left(1+\partial_{x}^2\right)^2u_1+2b u_0u_1-3u_0^2u_1\label{eq:SH}
\end{equation}
Since we have chosen $u_0$ to be stationary in time, $u_1$ must be a zero eigenvector of the SHE linearized about the solution a the saddle-node of the snaking branch.  Just as in Burke's calculation, the relevant eigenvector is the one that corresponding to the direction that adds periods to the localized solution ($u_{\text{amp}}$)  and it conveniently has an eigenvalue of 0 since we are right on the saddle-node.  Thus we can write the correction in the form $u_1=a( \tau)u_{\text{amp}}$.
We must go on order  $\mathcal{O}(\epsilon^2)$ to determine $a$.  At this order, the equation is
\begin{equation}
\partial_{\tau} u_1=r_+u_2-\left(1+\partial_{x}^2\right)^2u_2+2b u_0u_2-3u_0^2u_2 +(\delta+\rho \sin\omega \tau )u_0 +bu_1^2- 3u_0 u_1^2 \label{eq:SH}
\end{equation}
Because the linear operator acting on $u_2$ is self-adjoint and $u_{\text{amp}}$ is in it's nullspace, we use this equation to obtain the following solvability condition that determines $a$.  
\begin{equation}
\alpha_1 \dot{a} = \alpha_2 (\delta+\rho\sin\omega  \tau)+\alpha_3 a^2,
\end{equation}
where
\begin{align}
\alpha_1&=\int_0^L u_{\text{amp}}(x)^2 dx\\ \nonumber
\alpha_2&=\int_0^L u_{\text{amp}}(x) u_0(x) dx\\ \nonumber
\alpha_3&=\int_0^L u_{\text{amp}}(x)^3(b- 3u_0(x)) dx
\end{align}

Under the proper rescaling of parameters, the equation becomes
\begin{equation}
 \dot{a} =  \delta+\rho\sin\omega  \tau+\alpha a^2.
\end{equation}
I was unable to obtain an analytic solution to this equation.  My first attempt to transform by $b=1/a$ lead to the following equation,
\begin{equation}
 \dot{b} =  -(\delta+\rho\sin\omega  \tau) b^2-\alpha .
\end{equation}
A more general transformation of the form $a=b^n\tau^m$ also did not result in an equation of Bernoulli type or another form I could solve analytically.  I also tried the transformation $a= b \tan(\theta)$, under the assumption that $\alpha>0$ with no luck.  

\subsection{large oscillations}
We can also attempt this calculation for oscillations of order $r_0$, just outside of the stable region of the periodically forced case (e.g. $r_0\rightarrow r_0+\delta$).  This region is within the pinning region of the constant forcing case, with "sweet spots" for certain oscillation periods.  Here, we will assume that the nucleation events happen on a much slower time scale ($\tau=\epsilon t$) than the period of oscillation of the forcing.  The equation becomes
\begin{equation}
u_t+ \epsilon u_{\tau}=(r_0+\epsilon^2 \delta + \rho \sin\omega t) u-\left(1+\partial_{x}^2\right)^2u+b u^2-u^3 \label{eq:SH}.
\end{equation}
Again, writing $u$ as a power series in $\epsilon$ and expanding order by order, we try to find the dynamics of the nucleation process in this limit.  At leading order, we get
\begin{equation}
\partial_t u_0=(r_0 + \rho \sin\omega t) u_0-\left(1+\partial_{x}^2\right)^2u_0+b u_0^2-u_0^3 \label{eq:SHfrontspeedlosc0}.
\end{equation}
We can numerically calculate the orbit of $u_0(x,t)$, and will assume it is stationary on long times.  We may further like to assume that $r_0$ is at the right edge of the stable region so that we can assume the solution is marginally stable to nucleating periods when averaged over oscillations.  At order $\mathcal{O}(\epsilon)$, the equation is
 \begin{equation}
\partial_{\tau} u_0=\left[(r_0 + \rho \sin\omega t) -\left(1+\partial_{x}^2\right)^2+2 b u_0-3u_0^2 -\partial_t\right] u_1 = 0 \label{eq:SH}.
\end{equation} 
Thus $u_1$ is in the nullspace of the time-dependent linear operator $L$ defined in brackets above, it is the solution to the linearization of Eq.~\ref{eq:SHfrontspeedlosc0} about $u_0$. Now, if we do assume we are at the edge of a stable region, the eigenvector that points along the direction of nucleating periods on average ($u_a(x,t)$) will be marginal and this is the solution we want to write $u_1$ in terms of. 
 \begin{equation}
u_1(x,t,\tau)=a(\tau) u_a(x,t)
\end{equation}
We must go on to $\mathcal{O}(\epsilon^2)$ to find the condition that determines $a$.  At this order, the equation becomes
 \begin{equation}
\partial_{\tau} u_1=\left[(r_0 + \rho \sin\omega t) -\left(1+\partial_{x}^2\right)^2+2 b u_0-3u_0^2 -\partial_t\right] u_2 +\delta u_0 +bu_1^2-3u_0u_1^2 \label{eq:SH}.
\end{equation} 
Because the operator in brackets is not self-adjoint, we must modify Burke's procedure slightly here in order to take advantage of the Fredholm Alternative to derive a solvability condition.  We can define $v_a(x,t)$ as the solution to the adjoint problem:
 \begin{equation}
\left[(r_0 + \rho \sin\omega t) -\left(1+\partial_{x}^2\right)^2+2 b u_0-3u_0^2 +\partial_t\right] v_a = 0 \label{eq:SH}.
\end{equation} 
We note that taking the transformation $t\rightarrow -t$ and $\rho \rightarrow -\rho$ in Eq.~\ref{eq:SHfrontspeedlosc0} and then linearizing will produce the adjoint. Thus if $u_0$ is a solution that nucleates and then decays two periods over one cycle, then $v_a$ will be related to the eigenvector associated with the solution that decays two periods and then nucleates two periods?

Once we have $v_a$ and $u_a$, we can write down and solve the equation for $a$ in order to find the nucleation time.  The resulting equation has the same form as the one Burke derived originally, just with different coefficients.
\begin{equation}
 \alpha_1'\dot{a} =  \delta\alpha_2'+\alpha_3' a^2.
\end{equation}
The coefficients are integrals over space and time.
\begin{align}
\alpha_1'&=\int_0^{2\pi/\omega} \int_0^L v_a(x,t)u_a(x,t) dx dt\\ \nonumber
\alpha_2'&=\int_0^{2\pi/\omega} \int_0^L v_a(x,t) u_0(x,t) dxdt\\ \nonumber
\alpha_3'&=\int_0^{2\pi/\omega} \int_0^L v_a(x,t)u_a(x,t)^2(b- 3u_0(x,t)) dxdt
\end{align}
Assuming that this analysis is correct, it seems like the nucleation time should follow the same square root proportionality on $\delta$ that Burke found in the constant forcing case.


\bibliography{TimeForcingSHE_bibliography}



\end{document}




